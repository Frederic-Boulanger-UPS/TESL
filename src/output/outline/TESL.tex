%
\begin{isabellebody}%
\setisabellecontext{TESL}%
%
\begin{isamarkuptext}%
\chapter[Core TESL: Syntax and Basics]{The Core of the TESL Language: Syntax and Basics}%
\end{isamarkuptext}\isamarkuptrue%
%
\isadelimtheory
%
\endisadelimtheory
%
\isatagtheory
\isacommand{theory}\isamarkupfalse%
\ TESL\isanewline
\isakeyword{imports}\ Main\isanewline
\isanewline
\isakeyword{begin}%
\endisatagtheory
{\isafoldtheory}%
%
\isadelimtheory
%
\endisadelimtheory
%
\isadelimdocument
%
\endisadelimdocument
%
\isatagdocument
%
\isamarkupsection{Syntactic Representation%
}
\isamarkuptrue%
%
\endisatagdocument
{\isafolddocument}%
%
\isadelimdocument
%
\endisadelimdocument
%
\begin{isamarkuptext}%
We define here the syntax of TESL specifications.%
\end{isamarkuptext}\isamarkuptrue%
%
\isadelimdocument
%
\endisadelimdocument
%
\isatagdocument
%
\isamarkupsubsection{Basic elements of a specification%
}
\isamarkuptrue%
%
\endisatagdocument
{\isafolddocument}%
%
\isadelimdocument
%
\endisadelimdocument
%
\begin{isamarkuptext}%
The following items appear in specifications:

%
\begin{itemize}%
\item Clocks, which are identified by a name.

\item Tag constants are just constants of a type which denotes the metric time space.%
\end{itemize}%
\end{isamarkuptext}\isamarkuptrue%
\isacommand{datatype}\isamarkupfalse%
\ \ \ \ \ clock\ \ \ \ \ \ \ \ \ {\isacharequal}\ Clk\ {\isacartoucheopen}string{\isacartoucheclose}\isanewline
\isacommand{type{\isacharunderscore}synonym}\isamarkupfalse%
\ instant{\isacharunderscore}index\ {\isacharequal}\ {\isacartoucheopen}nat{\isacartoucheclose}\isanewline
\isanewline
\isacommand{datatype}\isamarkupfalse%
\ \ \ \ \ {\isacharprime}{\isasymtau}\ tag{\isacharunderscore}const\ {\isacharequal}\ \ TConst\ \ \ {\isacharparenleft}the{\isacharunderscore}tag{\isacharunderscore}const\ {\isacharcolon}\ {\isacharprime}{\isasymtau}{\isacharparenright}\ \ \ \ \ \ \ \ \ {\isacharparenleft}{\isachardoublequoteopen}{\isasymtau}\isactrlsub c\isactrlsub s\isactrlsub t{\isachardoublequoteclose}{\isacharparenright}%
\isadelimdocument
%
\endisadelimdocument
%
\isatagdocument
%
\isamarkupsubsection{Operators for the TESL language%
}
\isamarkuptrue%
%
\endisatagdocument
{\isafolddocument}%
%
\isadelimdocument
%
\endisadelimdocument
%
\begin{isamarkuptext}%
The type of atomic TESL constraints, which can be combined to form specifications.%
\end{isamarkuptext}\isamarkuptrue%
\isacommand{datatype}\isamarkupfalse%
\ {\isacharprime}{\isasymtau}\ TESL{\isacharunderscore}atomic\ {\isacharequal}\isanewline
\ \ \ \ SporadicOn\ \ \ \ \ \ \ {\isacartoucheopen}clock{\isacartoucheclose}\ {\isacartoucheopen}{\isacharprime}{\isasymtau}\ tag{\isacharunderscore}const{\isacartoucheclose}\ \ {\isacartoucheopen}clock{\isacartoucheclose}\ \ {\isacharparenleft}{\isachardoublequoteopen}{\isacharunderscore}\ sporadic\ {\isacharunderscore}\ on\ {\isacharunderscore}{\isachardoublequoteclose}\ {\isadigit{5}}{\isadigit{5}}{\isacharparenright}\isanewline
\ \ {\isacharbar}\ TagRelation\ \ \ \ \ \ {\isacartoucheopen}clock{\isacartoucheclose}\ {\isacartoucheopen}clock{\isacartoucheclose}\ {\isacartoucheopen}{\isacharparenleft}{\isacharprime}{\isasymtau}\ tag{\isacharunderscore}const\ {\isasymtimes}\ {\isacharprime}{\isasymtau}\ tag{\isacharunderscore}const{\isacharparenright}\ {\isasymRightarrow}\ bool{\isacartoucheclose}\ \isanewline
\ \ \ \ \ \ \ \ \ \ \ \ \ \ \ \ \ \ \ \ \ \ \ \ \ \ \ \ \ \ \ \ \ \ \ \ \ \ \ \ \ \ \ \ \ \ \ \ \ \ \ \ \ \ {\isacharparenleft}{\isachardoublequoteopen}time{\isacharminus}relation\ {\isasymlfloor}{\isacharunderscore}{\isacharcomma}\ {\isacharunderscore}{\isasymrfloor}\ {\isasymin}\ {\isacharunderscore}{\isachardoublequoteclose}\ {\isadigit{5}}{\isadigit{5}}{\isacharparenright}\isanewline
\ \ {\isacharbar}\ Implies\ \ \ \ \ \ \ \ \ \ {\isacartoucheopen}clock{\isacartoucheclose}\ {\isacartoucheopen}clock{\isacartoucheclose}\ \ \ \ \ \ \ \ \ \ \ \ \ \ \ \ \ \ {\isacharparenleft}\isakeyword{infixr}\ {\isachardoublequoteopen}implies{\isachardoublequoteclose}\ {\isadigit{5}}{\isadigit{5}}{\isacharparenright}\isanewline
\ \ {\isacharbar}\ ImpliesNot\ \ \ \ \ \ \ {\isacartoucheopen}clock{\isacartoucheclose}\ {\isacartoucheopen}clock{\isacartoucheclose}\ \ \ \ \ \ \ \ \ \ \ \ \ \ \ \ \ \ {\isacharparenleft}\isakeyword{infixr}\ {\isachardoublequoteopen}implies\ not{\isachardoublequoteclose}\ {\isadigit{5}}{\isadigit{5}}{\isacharparenright}\isanewline
\ \ {\isacharbar}\ TimeDelayedBy\ \ \ \ {\isacartoucheopen}clock{\isacartoucheclose}\ {\isacartoucheopen}{\isacharprime}{\isasymtau}\ tag{\isacharunderscore}const{\isacartoucheclose}\ {\isacartoucheopen}clock{\isacartoucheclose}\ {\isacartoucheopen}clock{\isacartoucheclose}\ \isanewline
\ \ \ \ \ \ \ \ \ \ \ \ \ \ \ \ \ \ \ \ \ \ \ \ \ \ \ \ \ \ \ \ \ \ \ \ \ \ \ \ \ \ \ \ \ \ \ \ \ \ \ \ \ \ {\isacharparenleft}{\isachardoublequoteopen}{\isacharunderscore}\ time{\isacharminus}delayed\ by\ {\isacharunderscore}\ on\ {\isacharunderscore}\ implies\ {\isacharunderscore}{\isachardoublequoteclose}\ {\isadigit{5}}{\isadigit{5}}{\isacharparenright}\isanewline
\ \ {\isacharbar}\ WeaklyPrecedes\ \ \ {\isacartoucheopen}clock{\isacartoucheclose}\ {\isacartoucheopen}clock{\isacartoucheclose}\ \ \ \ \ \ \ \ \ \ \ \ \ \ \ \ \ \ {\isacharparenleft}\isakeyword{infixr}\ {\isachardoublequoteopen}weakly\ precedes{\isachardoublequoteclose}\ {\isadigit{5}}{\isadigit{5}}{\isacharparenright}\isanewline
\ \ {\isacharbar}\ StrictlyPrecedes\ {\isacartoucheopen}clock{\isacartoucheclose}\ {\isacartoucheopen}clock{\isacartoucheclose}\ \ \ \ \ \ \ \ \ \ \ \ \ \ \ \ \ \ {\isacharparenleft}\isakeyword{infixr}\ {\isachardoublequoteopen}strictly\ precedes{\isachardoublequoteclose}\ {\isadigit{5}}{\isadigit{5}}{\isacharparenright}\isanewline
\ \ {\isacharbar}\ Kills\ \ \ \ \ \ \ \ \ \ \ \ {\isacartoucheopen}clock{\isacartoucheclose}\ {\isacartoucheopen}clock{\isacartoucheclose}\ \ \ \ \ \ \ \ \ \ \ \ \ \ \ \ \ \ {\isacharparenleft}\isakeyword{infixr}\ {\isachardoublequoteopen}kills{\isachardoublequoteclose}\ {\isadigit{5}}{\isadigit{5}}{\isacharparenright}%
\begin{isamarkuptext}%
A TESL formula is just a list of atomic constraints, with implicit conjunction
  for the semantics.%
\end{isamarkuptext}\isamarkuptrue%
\isacommand{type{\isacharunderscore}synonym}\isamarkupfalse%
\ {\isacharprime}{\isasymtau}\ TESL{\isacharunderscore}formula\ {\isacharequal}\ {\isacartoucheopen}{\isacharprime}{\isasymtau}\ TESL{\isacharunderscore}atomic\ list{\isacartoucheclose}%
\begin{isamarkuptext}%
We call \emph{positive atoms} the atomic constraints that create ticks from nothing.
  Only sporadic constraints are positive in the current version of TESL.%
\end{isamarkuptext}\isamarkuptrue%
\isacommand{fun}\isamarkupfalse%
\ positive{\isacharunderscore}atom\ {\isacharcolon}{\isacharcolon}\ {\isacartoucheopen}{\isacharprime}{\isasymtau}\ TESL{\isacharunderscore}atomic\ {\isasymRightarrow}\ bool{\isacartoucheclose}\ \isakeyword{where}\isanewline
\ \ \ \ {\isacartoucheopen}positive{\isacharunderscore}atom\ {\isacharparenleft}{\isacharunderscore}\ sporadic\ {\isacharunderscore}\ on\ {\isacharunderscore}{\isacharparenright}\ {\isacharequal}\ True{\isacartoucheclose}\isanewline
\ \ {\isacharbar}\ {\isacartoucheopen}positive{\isacharunderscore}atom\ {\isacharunderscore}\ \ \ \ \ \ \ \ \ \ \ \ \ \ \ \ \ \ \ {\isacharequal}\ False{\isacartoucheclose}%
\begin{isamarkuptext}%
The \isa{NoSporadic} function removes sporadic constraints from a TESL formula.%
\end{isamarkuptext}\isamarkuptrue%
\isacommand{abbreviation}\isamarkupfalse%
\ NoSporadic\ {\isacharcolon}{\isacharcolon}\ {\isacartoucheopen}{\isacharprime}{\isasymtau}\ TESL{\isacharunderscore}formula\ {\isasymRightarrow}\ {\isacharprime}{\isasymtau}\ TESL{\isacharunderscore}formula{\isacartoucheclose}\isanewline
\isakeyword{where}\ \isanewline
\ \ {\isacartoucheopen}NoSporadic\ f\ {\isasymequiv}\ {\isacharparenleft}List{\isachardot}filter\ {\isacharparenleft}{\isasymlambda}f\isactrlsub a\isactrlsub t\isactrlsub o\isactrlsub m{\isachardot}\ case\ f\isactrlsub a\isactrlsub t\isactrlsub o\isactrlsub m\ of\isanewline
\ \ \ \ \ \ {\isacharunderscore}\ sporadic\ {\isacharunderscore}\ on\ {\isacharunderscore}\ {\isasymRightarrow}\ False\isanewline
\ \ \ \ {\isacharbar}\ {\isacharunderscore}\ {\isasymRightarrow}\ True{\isacharparenright}\ f{\isacharparenright}{\isacartoucheclose}%
\isadelimdocument
%
\endisadelimdocument
%
\isatagdocument
%
\isamarkupsubsection{Field Structure of the Metric Time Space%
}
\isamarkuptrue%
%
\endisatagdocument
{\isafolddocument}%
%
\isadelimdocument
%
\endisadelimdocument
%
\begin{isamarkuptext}%
In order to handle tag relations and delays, tags must belong to a field.
  We show here that this is the case when the type parameter of \isa{{\isacharprime}{\isasymtau}\ tag{\isacharunderscore}const} 
  is itself a field.%
\end{isamarkuptext}\isamarkuptrue%
\isacommand{instantiation}\isamarkupfalse%
\ tag{\isacharunderscore}const\ {\isacharcolon}{\isacharcolon}{\isacharparenleft}field{\isacharparenright}field\isanewline
\isakeyword{begin}\isanewline
\ \ \isacommand{fun}\isamarkupfalse%
\ inverse{\isacharunderscore}tag{\isacharunderscore}const\isanewline
\ \ \isakeyword{where}\ {\isacartoucheopen}inverse\ {\isacharparenleft}{\isasymtau}\isactrlsub c\isactrlsub s\isactrlsub t\ t{\isacharparenright}\ {\isacharequal}\ {\isasymtau}\isactrlsub c\isactrlsub s\isactrlsub t\ {\isacharparenleft}inverse\ t{\isacharparenright}{\isacartoucheclose}\isanewline
\isanewline
\ \ \isacommand{fun}\isamarkupfalse%
\ divide{\isacharunderscore}tag{\isacharunderscore}const\ \isanewline
\ \ \ \ \isakeyword{where}\ {\isacartoucheopen}divide\ {\isacharparenleft}{\isasymtau}\isactrlsub c\isactrlsub s\isactrlsub t\ t\isactrlsub {\isadigit{1}}{\isacharparenright}\ {\isacharparenleft}{\isasymtau}\isactrlsub c\isactrlsub s\isactrlsub t\ t\isactrlsub {\isadigit{2}}{\isacharparenright}\ {\isacharequal}\ {\isasymtau}\isactrlsub c\isactrlsub s\isactrlsub t\ {\isacharparenleft}divide\ t\isactrlsub {\isadigit{1}}\ t\isactrlsub {\isadigit{2}}{\isacharparenright}{\isacartoucheclose}\isanewline
\isanewline
\ \ \isacommand{fun}\isamarkupfalse%
\ uminus{\isacharunderscore}tag{\isacharunderscore}const\isanewline
\ \ \ \ \isakeyword{where}\ {\isacartoucheopen}uminus\ {\isacharparenleft}{\isasymtau}\isactrlsub c\isactrlsub s\isactrlsub t\ t{\isacharparenright}\ {\isacharequal}\ {\isasymtau}\isactrlsub c\isactrlsub s\isactrlsub t\ {\isacharparenleft}uminus\ t{\isacharparenright}{\isacartoucheclose}\isanewline
\isanewline
\isacommand{fun}\isamarkupfalse%
\ minus{\isacharunderscore}tag{\isacharunderscore}const\isanewline
\ \ \isakeyword{where}\ {\isacartoucheopen}minus\ {\isacharparenleft}{\isasymtau}\isactrlsub c\isactrlsub s\isactrlsub t\ t\isactrlsub {\isadigit{1}}{\isacharparenright}\ {\isacharparenleft}{\isasymtau}\isactrlsub c\isactrlsub s\isactrlsub t\ t\isactrlsub {\isadigit{2}}{\isacharparenright}\ {\isacharequal}\ {\isasymtau}\isactrlsub c\isactrlsub s\isactrlsub t\ {\isacharparenleft}minus\ t\isactrlsub {\isadigit{1}}\ t\isactrlsub {\isadigit{2}}{\isacharparenright}{\isacartoucheclose}\isanewline
\isanewline
\isacommand{definition}\isamarkupfalse%
\ {\isacartoucheopen}one{\isacharunderscore}tag{\isacharunderscore}const\ {\isasymequiv}\ {\isasymtau}\isactrlsub c\isactrlsub s\isactrlsub t\ {\isadigit{1}}{\isacartoucheclose}\isanewline
\isanewline
\isacommand{fun}\isamarkupfalse%
\ times{\isacharunderscore}tag{\isacharunderscore}const\isanewline
\ \ \isakeyword{where}\ {\isacartoucheopen}times\ {\isacharparenleft}{\isasymtau}\isactrlsub c\isactrlsub s\isactrlsub t\ t\isactrlsub {\isadigit{1}}{\isacharparenright}\ {\isacharparenleft}{\isasymtau}\isactrlsub c\isactrlsub s\isactrlsub t\ t\isactrlsub {\isadigit{2}}{\isacharparenright}\ {\isacharequal}\ {\isasymtau}\isactrlsub c\isactrlsub s\isactrlsub t\ {\isacharparenleft}times\ t\isactrlsub {\isadigit{1}}\ t\isactrlsub {\isadigit{2}}{\isacharparenright}{\isacartoucheclose}\isanewline
\isanewline
\isacommand{definition}\isamarkupfalse%
\ {\isacartoucheopen}zero{\isacharunderscore}tag{\isacharunderscore}const\ {\isasymequiv}\ {\isasymtau}\isactrlsub c\isactrlsub s\isactrlsub t\ {\isadigit{0}}{\isacartoucheclose}\isanewline
\isanewline
\isacommand{fun}\isamarkupfalse%
\ plus{\isacharunderscore}tag{\isacharunderscore}const\isanewline
\ \ \isakeyword{where}\ {\isacartoucheopen}plus\ {\isacharparenleft}{\isasymtau}\isactrlsub c\isactrlsub s\isactrlsub t\ t\isactrlsub {\isadigit{1}}{\isacharparenright}\ {\isacharparenleft}{\isasymtau}\isactrlsub c\isactrlsub s\isactrlsub t\ t\isactrlsub {\isadigit{2}}{\isacharparenright}\ {\isacharequal}\ {\isasymtau}\isactrlsub c\isactrlsub s\isactrlsub t\ {\isacharparenleft}plus\ t\isactrlsub {\isadigit{1}}\ t\isactrlsub {\isadigit{2}}{\isacharparenright}{\isacartoucheclose}\isanewline
\isanewline
\isacommand{instance}\isamarkupfalse%
%
\isadelimproof
\ %
\endisadelimproof
%
\isatagproof
\isacommand{proof}\isamarkupfalse%
%
\begin{isamarkuptext}%
Multiplication is associative.%
\end{isamarkuptext}\isamarkuptrue%
\ \ \isacommand{fix}\isamarkupfalse%
\ a{\isacharcolon}{\isacharcolon}{\isacartoucheopen}{\isacharprime}{\isasymtau}{\isacharcolon}{\isacharcolon}field\ tag{\isacharunderscore}const{\isacartoucheclose}\ \isakeyword{and}\ b{\isacharcolon}{\isacharcolon}{\isacartoucheopen}{\isacharprime}{\isasymtau}{\isacharcolon}{\isacharcolon}field\ tag{\isacharunderscore}const{\isacartoucheclose}\isanewline
\ \ \ \ \ \ \ \ \ \ \ \ \ \ \ \ \ \ \ \ \ \ \ \ \ \ \ \ \ \ \ \isakeyword{and}\ c{\isacharcolon}{\isacharcolon}{\isacartoucheopen}{\isacharprime}{\isasymtau}{\isacharcolon}{\isacharcolon}field\ tag{\isacharunderscore}const{\isacartoucheclose}\isanewline
\ \ \isacommand{obtain}\isamarkupfalse%
\ u\ v\ w\ \isakeyword{where}\ {\isacartoucheopen}a\ {\isacharequal}\ {\isasymtau}\isactrlsub c\isactrlsub s\isactrlsub t\ u{\isacartoucheclose}\ \isakeyword{and}\ {\isacartoucheopen}b\ {\isacharequal}\ {\isasymtau}\isactrlsub c\isactrlsub s\isactrlsub t\ v{\isacartoucheclose}\ \isakeyword{and}\ {\isacartoucheopen}c\ {\isacharequal}\ {\isasymtau}\isactrlsub c\isactrlsub s\isactrlsub t\ w{\isacartoucheclose}\isanewline
\ \ \ \ \isacommand{using}\isamarkupfalse%
\ tag{\isacharunderscore}const{\isachardot}exhaust\ \isacommand{by}\isamarkupfalse%
\ metis\isanewline
\ \ \isacommand{thus}\isamarkupfalse%
\ {\isacartoucheopen}a\ {\isacharasterisk}\ b\ {\isacharasterisk}\ c\ {\isacharequal}\ a\ {\isacharasterisk}\ {\isacharparenleft}b\ {\isacharasterisk}\ c{\isacharparenright}{\isacartoucheclose}\isanewline
\ \ \ \ \isacommand{by}\isamarkupfalse%
\ {\isacharparenleft}simp\ add{\isacharcolon}\ TESL{\isachardot}times{\isacharunderscore}tag{\isacharunderscore}const{\isachardot}simps{\isacharparenright}\isanewline
\isacommand{next}\isamarkupfalse%
%
\begin{isamarkuptext}%
Multiplication is commutative.%
\end{isamarkuptext}\isamarkuptrue%
\ \ \isacommand{fix}\isamarkupfalse%
\ a{\isacharcolon}{\isacharcolon}{\isacartoucheopen}{\isacharprime}{\isasymtau}{\isacharcolon}{\isacharcolon}field\ tag{\isacharunderscore}const{\isacartoucheclose}\ \isakeyword{and}\ b{\isacharcolon}{\isacharcolon}{\isacartoucheopen}{\isacharprime}{\isasymtau}{\isacharcolon}{\isacharcolon}field\ tag{\isacharunderscore}const{\isacartoucheclose}\isanewline
\ \ \isacommand{obtain}\isamarkupfalse%
\ u\ v\ \isakeyword{where}\ {\isacartoucheopen}a\ {\isacharequal}\ {\isasymtau}\isactrlsub c\isactrlsub s\isactrlsub t\ u{\isacartoucheclose}\ \isakeyword{and}\ {\isacartoucheopen}b\ {\isacharequal}\ {\isasymtau}\isactrlsub c\isactrlsub s\isactrlsub t\ v{\isacartoucheclose}\ \isacommand{using}\isamarkupfalse%
\ tag{\isacharunderscore}const{\isachardot}exhaust\ \isacommand{by}\isamarkupfalse%
\ metis\isanewline
\ \ \isacommand{thus}\isamarkupfalse%
\ {\isacartoucheopen}\ a\ {\isacharasterisk}\ b\ {\isacharequal}\ b\ {\isacharasterisk}\ a{\isacartoucheclose}\isanewline
\ \ \ \ \isacommand{by}\isamarkupfalse%
\ {\isacharparenleft}simp\ add{\isacharcolon}\ TESL{\isachardot}times{\isacharunderscore}tag{\isacharunderscore}const{\isachardot}simps{\isacharparenright}\isanewline
\isacommand{next}\isamarkupfalse%
%
\begin{isamarkuptext}%
One is neutral for multiplication.%
\end{isamarkuptext}\isamarkuptrue%
\ \ \isacommand{fix}\isamarkupfalse%
\ a{\isacharcolon}{\isacharcolon}{\isacartoucheopen}{\isacharprime}{\isasymtau}{\isacharcolon}{\isacharcolon}field\ tag{\isacharunderscore}const{\isacartoucheclose}\isanewline
\ \ \isacommand{obtain}\isamarkupfalse%
\ u\ \isakeyword{where}\ {\isacartoucheopen}a\ {\isacharequal}\ {\isasymtau}\isactrlsub c\isactrlsub s\isactrlsub t\ u{\isacartoucheclose}\ \isacommand{using}\isamarkupfalse%
\ tag{\isacharunderscore}const{\isachardot}exhaust\ \isacommand{by}\isamarkupfalse%
\ blast\isanewline
\ \ \isacommand{thus}\isamarkupfalse%
\ {\isacartoucheopen}{\isadigit{1}}\ {\isacharasterisk}\ a\ {\isacharequal}\ a{\isacartoucheclose}\isanewline
\ \ \ \ \isacommand{by}\isamarkupfalse%
\ {\isacharparenleft}simp\ add{\isacharcolon}\ TESL{\isachardot}times{\isacharunderscore}tag{\isacharunderscore}const{\isachardot}simps\ one{\isacharunderscore}tag{\isacharunderscore}const{\isacharunderscore}def{\isacharparenright}\isanewline
\isacommand{next}\isamarkupfalse%
%
\begin{isamarkuptext}%
Addition is associative.%
\end{isamarkuptext}\isamarkuptrue%
\ \ \isacommand{fix}\isamarkupfalse%
\ a{\isacharcolon}{\isacharcolon}{\isacartoucheopen}{\isacharprime}{\isasymtau}{\isacharcolon}{\isacharcolon}field\ tag{\isacharunderscore}const{\isacartoucheclose}\ \isakeyword{and}\ b{\isacharcolon}{\isacharcolon}{\isacartoucheopen}{\isacharprime}{\isasymtau}{\isacharcolon}{\isacharcolon}field\ tag{\isacharunderscore}const{\isacartoucheclose}\isanewline
\ \ \ \ \ \ \ \ \ \ \ \ \ \ \ \ \ \ \ \ \ \ \ \ \ \ \ \ \ \ \ \isakeyword{and}\ c{\isacharcolon}{\isacharcolon}{\isacartoucheopen}{\isacharprime}{\isasymtau}{\isacharcolon}{\isacharcolon}field\ tag{\isacharunderscore}const{\isacartoucheclose}\isanewline
\ \ \isacommand{obtain}\isamarkupfalse%
\ u\ v\ w\ \isakeyword{where}\ {\isacartoucheopen}a\ {\isacharequal}\ {\isasymtau}\isactrlsub c\isactrlsub s\isactrlsub t\ u{\isacartoucheclose}\ \isakeyword{and}\ {\isacartoucheopen}b\ {\isacharequal}\ {\isasymtau}\isactrlsub c\isactrlsub s\isactrlsub t\ v{\isacartoucheclose}\ \isakeyword{and}\ {\isacartoucheopen}c\ {\isacharequal}\ {\isasymtau}\isactrlsub c\isactrlsub s\isactrlsub t\ w{\isacartoucheclose}\isanewline
\ \ \ \ \isacommand{using}\isamarkupfalse%
\ tag{\isacharunderscore}const{\isachardot}exhaust\ \isacommand{by}\isamarkupfalse%
\ metis\isanewline
\ \ \isacommand{thus}\isamarkupfalse%
\ {\isacartoucheopen}a\ {\isacharplus}\ b\ {\isacharplus}\ c\ {\isacharequal}\ a\ {\isacharplus}\ {\isacharparenleft}b\ {\isacharplus}\ c{\isacharparenright}{\isacartoucheclose}\isanewline
\ \ \ \ \isacommand{by}\isamarkupfalse%
\ {\isacharparenleft}simp\ add{\isacharcolon}\ TESL{\isachardot}plus{\isacharunderscore}tag{\isacharunderscore}const{\isachardot}simps{\isacharparenright}\isanewline
\isacommand{next}\isamarkupfalse%
%
\begin{isamarkuptext}%
Addition is commutative.%
\end{isamarkuptext}\isamarkuptrue%
\ \ \isacommand{fix}\isamarkupfalse%
\ a{\isacharcolon}{\isacharcolon}{\isacartoucheopen}{\isacharprime}{\isasymtau}{\isacharcolon}{\isacharcolon}field\ tag{\isacharunderscore}const{\isacartoucheclose}\ \isakeyword{and}\ b{\isacharcolon}{\isacharcolon}{\isacartoucheopen}{\isacharprime}{\isasymtau}{\isacharcolon}{\isacharcolon}field\ tag{\isacharunderscore}const{\isacartoucheclose}\isanewline
\ \ \isacommand{obtain}\isamarkupfalse%
\ u\ v\ \isakeyword{where}\ {\isacartoucheopen}a\ {\isacharequal}\ {\isasymtau}\isactrlsub c\isactrlsub s\isactrlsub t\ u{\isacartoucheclose}\ \isakeyword{and}\ {\isacartoucheopen}b\ {\isacharequal}\ {\isasymtau}\isactrlsub c\isactrlsub s\isactrlsub t\ v{\isacartoucheclose}\ \isacommand{using}\isamarkupfalse%
\ tag{\isacharunderscore}const{\isachardot}exhaust\ \isacommand{by}\isamarkupfalse%
\ metis\isanewline
\ \ \isacommand{thus}\isamarkupfalse%
\ {\isacartoucheopen}a\ {\isacharplus}\ b\ {\isacharequal}\ b\ {\isacharplus}\ a{\isacartoucheclose}\isanewline
\ \ \ \ \isacommand{by}\isamarkupfalse%
\ {\isacharparenleft}simp\ add{\isacharcolon}\ TESL{\isachardot}plus{\isacharunderscore}tag{\isacharunderscore}const{\isachardot}simps{\isacharparenright}\isanewline
\isacommand{next}\isamarkupfalse%
%
\begin{isamarkuptext}%
Zero is neutral for addition.%
\end{isamarkuptext}\isamarkuptrue%
\ \ \isacommand{fix}\isamarkupfalse%
\ a{\isacharcolon}{\isacharcolon}{\isacartoucheopen}{\isacharprime}{\isasymtau}{\isacharcolon}{\isacharcolon}field\ tag{\isacharunderscore}const{\isacartoucheclose}\isanewline
\ \ \isacommand{obtain}\isamarkupfalse%
\ u\ \isakeyword{where}\ {\isacartoucheopen}a\ {\isacharequal}\ {\isasymtau}\isactrlsub c\isactrlsub s\isactrlsub t\ u{\isacartoucheclose}\ \isacommand{using}\isamarkupfalse%
\ tag{\isacharunderscore}const{\isachardot}exhaust\ \isacommand{by}\isamarkupfalse%
\ blast\isanewline
\ \ \isacommand{thus}\isamarkupfalse%
\ {\isacartoucheopen}{\isadigit{0}}\ {\isacharplus}\ a\ {\isacharequal}\ a{\isacartoucheclose}\isanewline
\ \ \ \ \isacommand{by}\isamarkupfalse%
\ {\isacharparenleft}simp\ add{\isacharcolon}\ TESL{\isachardot}plus{\isacharunderscore}tag{\isacharunderscore}const{\isachardot}simps\ zero{\isacharunderscore}tag{\isacharunderscore}const{\isacharunderscore}def{\isacharparenright}\isanewline
\isacommand{next}\isamarkupfalse%
%
\begin{isamarkuptext}%
The sum of an element and its opposite is zero.%
\end{isamarkuptext}\isamarkuptrue%
\ \ \isacommand{fix}\isamarkupfalse%
\ a{\isacharcolon}{\isacharcolon}{\isacartoucheopen}{\isacharprime}{\isasymtau}{\isacharcolon}{\isacharcolon}field\ tag{\isacharunderscore}const{\isacartoucheclose}\isanewline
\ \ \isacommand{obtain}\isamarkupfalse%
\ u\ \isakeyword{where}\ {\isacartoucheopen}a\ {\isacharequal}\ {\isasymtau}\isactrlsub c\isactrlsub s\isactrlsub t\ u{\isacartoucheclose}\ \isacommand{using}\isamarkupfalse%
\ tag{\isacharunderscore}const{\isachardot}exhaust\ \isacommand{by}\isamarkupfalse%
\ blast\isanewline
\ \ \isacommand{thus}\isamarkupfalse%
\ {\isacartoucheopen}{\isacharminus}a\ {\isacharplus}\ a\ {\isacharequal}\ {\isadigit{0}}{\isacartoucheclose}\isanewline
\ \ \ \ \isacommand{by}\isamarkupfalse%
\ {\isacharparenleft}simp\ add{\isacharcolon}\ TESL{\isachardot}plus{\isacharunderscore}tag{\isacharunderscore}const{\isachardot}simps\isanewline
\ \ \ \ \ \ \ \ \ \ \ \ \ \ \ \ \ \ TESL{\isachardot}uminus{\isacharunderscore}tag{\isacharunderscore}const{\isachardot}simps\isanewline
\ \ \ \ \ \ \ \ \ \ \ \ \ \ \ \ \ \ zero{\isacharunderscore}tag{\isacharunderscore}const{\isacharunderscore}def{\isacharparenright}\isanewline
\isacommand{next}\isamarkupfalse%
%
\begin{isamarkuptext}%
Subtraction is adding the opposite.%
\end{isamarkuptext}\isamarkuptrue%
\ \ \isacommand{fix}\isamarkupfalse%
\ a{\isacharcolon}{\isacharcolon}{\isacartoucheopen}{\isacharprime}{\isasymtau}{\isacharcolon}{\isacharcolon}field\ tag{\isacharunderscore}const{\isacartoucheclose}\ \isakeyword{and}\ b{\isacharcolon}{\isacharcolon}{\isacartoucheopen}{\isacharprime}{\isasymtau}{\isacharcolon}{\isacharcolon}field\ tag{\isacharunderscore}const{\isacartoucheclose}\isanewline
\ \ \isacommand{obtain}\isamarkupfalse%
\ u\ v\ \isakeyword{where}\ {\isacartoucheopen}a\ {\isacharequal}\ {\isasymtau}\isactrlsub c\isactrlsub s\isactrlsub t\ u{\isacartoucheclose}\ \isakeyword{and}\ {\isacartoucheopen}b\ {\isacharequal}\ {\isasymtau}\isactrlsub c\isactrlsub s\isactrlsub t\ v{\isacartoucheclose}\ \isacommand{using}\isamarkupfalse%
\ tag{\isacharunderscore}const{\isachardot}exhaust\ \isacommand{by}\isamarkupfalse%
\ metis\isanewline
\ \ \isacommand{thus}\isamarkupfalse%
\ {\isacartoucheopen}a\ {\isacharminus}\ b\ {\isacharequal}\ a\ {\isacharplus}\ {\isacharminus}b{\isacartoucheclose}\isanewline
\ \ \ \ \isacommand{by}\isamarkupfalse%
\ {\isacharparenleft}simp\ add{\isacharcolon}\ TESL{\isachardot}minus{\isacharunderscore}tag{\isacharunderscore}const{\isachardot}simps\isanewline
\ \ \ \ \ \ \ \ \ \ \ \ \ \ \ \ \ \ TESL{\isachardot}plus{\isacharunderscore}tag{\isacharunderscore}const{\isachardot}simps\isanewline
\ \ \ \ \ \ \ \ \ \ \ \ \ \ \ \ \ \ TESL{\isachardot}uminus{\isacharunderscore}tag{\isacharunderscore}const{\isachardot}simps{\isacharparenright}\isanewline
\isacommand{next}\isamarkupfalse%
%
\begin{isamarkuptext}%
Distributive property of multiplication over addition.%
\end{isamarkuptext}\isamarkuptrue%
\ \ \isacommand{fix}\isamarkupfalse%
\ a{\isacharcolon}{\isacharcolon}{\isacartoucheopen}{\isacharprime}{\isasymtau}{\isacharcolon}{\isacharcolon}field\ tag{\isacharunderscore}const{\isacartoucheclose}\ \isakeyword{and}\ b{\isacharcolon}{\isacharcolon}{\isacartoucheopen}{\isacharprime}{\isasymtau}{\isacharcolon}{\isacharcolon}field\ tag{\isacharunderscore}const{\isacartoucheclose}\isanewline
\ \ \ \ \ \ \ \ \ \ \ \ \ \ \ \ \ \ \ \ \ \ \ \ \ \ \ \ \ \ \ \isakeyword{and}\ c{\isacharcolon}{\isacharcolon}{\isacartoucheopen}{\isacharprime}{\isasymtau}{\isacharcolon}{\isacharcolon}field\ tag{\isacharunderscore}const{\isacartoucheclose}\isanewline
\ \ \isacommand{obtain}\isamarkupfalse%
\ u\ v\ w\ \isakeyword{where}\ {\isacartoucheopen}a\ {\isacharequal}\ {\isasymtau}\isactrlsub c\isactrlsub s\isactrlsub t\ u{\isacartoucheclose}\ \isakeyword{and}\ {\isacartoucheopen}b\ {\isacharequal}\ {\isasymtau}\isactrlsub c\isactrlsub s\isactrlsub t\ v{\isacartoucheclose}\ \isakeyword{and}\ {\isacartoucheopen}c\ {\isacharequal}\ {\isasymtau}\isactrlsub c\isactrlsub s\isactrlsub t\ w{\isacartoucheclose}\isanewline
\ \ \ \ \isacommand{using}\isamarkupfalse%
\ tag{\isacharunderscore}const{\isachardot}exhaust\ \isacommand{by}\isamarkupfalse%
\ metis\isanewline
\ \ \isacommand{thus}\isamarkupfalse%
\ {\isacartoucheopen}{\isacharparenleft}a\ {\isacharplus}\ b{\isacharparenright}\ {\isacharasterisk}\ c\ {\isacharequal}\ a\ {\isacharasterisk}\ c\ {\isacharplus}\ b\ {\isacharasterisk}\ c{\isacartoucheclose}\isanewline
\ \ \ \ \isacommand{by}\isamarkupfalse%
\ {\isacharparenleft}simp\ add{\isacharcolon}\ TESL{\isachardot}plus{\isacharunderscore}tag{\isacharunderscore}const{\isachardot}simps\isanewline
\ \ \ \ \ \ \ \ \ \ \ \ \ \ \ \ \ \ TESL{\isachardot}times{\isacharunderscore}tag{\isacharunderscore}const{\isachardot}simps\isanewline
\ \ \ \ \ \ \ \ \ \ \ \ \ \ \ \ \ \ ring{\isacharunderscore}class{\isachardot}ring{\isacharunderscore}distribs{\isacharparenleft}{\isadigit{2}}{\isacharparenright}{\isacharparenright}\isanewline
\isacommand{next}\isamarkupfalse%
%
\begin{isamarkuptext}%
The neutral elements are distinct.%
\end{isamarkuptext}\isamarkuptrue%
\ \ \isacommand{show}\isamarkupfalse%
\ {\isacartoucheopen}{\isacharparenleft}{\isadigit{0}}{\isacharcolon}{\isacharcolon}{\isacharparenleft}{\isacharprime}{\isasymtau}{\isacharcolon}{\isacharcolon}field\ tag{\isacharunderscore}const{\isacharparenright}{\isacharparenright}\ {\isasymnoteq}\ {\isadigit{1}}{\isacartoucheclose}\isanewline
\ \ \ \ \isacommand{by}\isamarkupfalse%
\ {\isacharparenleft}simp\ add{\isacharcolon}\ one{\isacharunderscore}tag{\isacharunderscore}const{\isacharunderscore}def\ zero{\isacharunderscore}tag{\isacharunderscore}const{\isacharunderscore}def{\isacharparenright}\isanewline
\isacommand{next}\isamarkupfalse%
%
\begin{isamarkuptext}%
The product of an element and its inverse is 1.%
\end{isamarkuptext}\isamarkuptrue%
\ \ \isacommand{fix}\isamarkupfalse%
\ a{\isacharcolon}{\isacharcolon}{\isacartoucheopen}{\isacharprime}{\isasymtau}{\isacharcolon}{\isacharcolon}field\ tag{\isacharunderscore}const{\isacartoucheclose}\ \isacommand{assume}\isamarkupfalse%
\ h{\isacharcolon}{\isacartoucheopen}a\ {\isasymnoteq}\ {\isadigit{0}}{\isacartoucheclose}\isanewline
\ \ \isacommand{obtain}\isamarkupfalse%
\ u\ \isakeyword{where}\ {\isacartoucheopen}a\ {\isacharequal}\ {\isasymtau}\isactrlsub c\isactrlsub s\isactrlsub t\ u{\isacartoucheclose}\ \isacommand{using}\isamarkupfalse%
\ tag{\isacharunderscore}const{\isachardot}exhaust\ \isacommand{by}\isamarkupfalse%
\ blast\isanewline
\ \ \isacommand{moreover}\isamarkupfalse%
\ \isacommand{with}\isamarkupfalse%
\ h\ \isacommand{have}\isamarkupfalse%
\ {\isacartoucheopen}u\ {\isasymnoteq}\ {\isadigit{0}}{\isacartoucheclose}\ \isacommand{by}\isamarkupfalse%
\ {\isacharparenleft}simp\ add{\isacharcolon}\ zero{\isacharunderscore}tag{\isacharunderscore}const{\isacharunderscore}def{\isacharparenright}\isanewline
\ \ \isacommand{ultimately}\isamarkupfalse%
\ \isacommand{show}\isamarkupfalse%
\ {\isacartoucheopen}inverse\ a\ {\isacharasterisk}\ a\ {\isacharequal}\ {\isadigit{1}}{\isacartoucheclose}\isanewline
\ \ \ \ \isacommand{by}\isamarkupfalse%
\ {\isacharparenleft}simp\ add{\isacharcolon}\ TESL{\isachardot}inverse{\isacharunderscore}tag{\isacharunderscore}const{\isachardot}simps\isanewline
\ \ \ \ \ \ \ \ \ \ \ \ \ \ \ \ \ \ TESL{\isachardot}times{\isacharunderscore}tag{\isacharunderscore}const{\isachardot}simps\isanewline
\ \ \ \ \ \ \ \ \ \ \ \ \ \ \ \ \ \ one{\isacharunderscore}tag{\isacharunderscore}const{\isacharunderscore}def{\isacharparenright}\isanewline
\isacommand{next}\isamarkupfalse%
%
\begin{isamarkuptext}%
Dividing is multiplying by the inverse.%
\end{isamarkuptext}\isamarkuptrue%
\ \ \isacommand{fix}\isamarkupfalse%
\ a{\isacharcolon}{\isacharcolon}{\isacartoucheopen}{\isacharprime}{\isasymtau}{\isacharcolon}{\isacharcolon}field\ tag{\isacharunderscore}const{\isacartoucheclose}\ \isakeyword{and}\ b{\isacharcolon}{\isacharcolon}{\isacartoucheopen}{\isacharprime}{\isasymtau}{\isacharcolon}{\isacharcolon}field\ tag{\isacharunderscore}const{\isacartoucheclose}\isanewline
\ \ \isacommand{obtain}\isamarkupfalse%
\ u\ v\ \isakeyword{where}\ {\isacartoucheopen}a\ {\isacharequal}\ {\isasymtau}\isactrlsub c\isactrlsub s\isactrlsub t\ u{\isacartoucheclose}\ \isakeyword{and}\ {\isacartoucheopen}b\ {\isacharequal}\ {\isasymtau}\isactrlsub c\isactrlsub s\isactrlsub t\ v{\isacartoucheclose}\ \isacommand{using}\isamarkupfalse%
\ tag{\isacharunderscore}const{\isachardot}exhaust\ \isacommand{by}\isamarkupfalse%
\ metis\isanewline
\ \ \isacommand{thus}\isamarkupfalse%
\ {\isacartoucheopen}a\ div\ b\ {\isacharequal}\ a\ {\isacharasterisk}\ inverse\ b{\isacartoucheclose}\isanewline
\ \ \ \ \isacommand{by}\isamarkupfalse%
\ {\isacharparenleft}simp\ add{\isacharcolon}\ TESL{\isachardot}divide{\isacharunderscore}tag{\isacharunderscore}const{\isachardot}simps\isanewline
\ \ \ \ \ \ \ \ \ \ \ \ \ \ \ \ \ \ TESL{\isachardot}inverse{\isacharunderscore}tag{\isacharunderscore}const{\isachardot}simps\isanewline
\ \ \ \ \ \ \ \ \ \ \ \ \ \ \ \ \ \ TESL{\isachardot}times{\isacharunderscore}tag{\isacharunderscore}const{\isachardot}simps\isanewline
\ \ \ \ \ \ \ \ \ \ \ \ \ \ \ \ \ \ divide{\isacharunderscore}inverse{\isacharparenright}\isanewline
\isacommand{next}\isamarkupfalse%
%
\begin{isamarkuptext}%
Zero is its own inverse.%
\end{isamarkuptext}\isamarkuptrue%
\ \ \isacommand{show}\isamarkupfalse%
\ {\isacartoucheopen}inverse\ {\isacharparenleft}{\isadigit{0}}{\isacharcolon}{\isacharcolon}{\isacharparenleft}{\isacharprime}{\isasymtau}{\isacharcolon}{\isacharcolon}field\ tag{\isacharunderscore}const{\isacharparenright}{\isacharparenright}\ {\isacharequal}\ {\isadigit{0}}{\isacartoucheclose}\isanewline
\ \ \ \ \isacommand{by}\isamarkupfalse%
\ {\isacharparenleft}simp\ add{\isacharcolon}\ TESL{\isachardot}inverse{\isacharunderscore}tag{\isacharunderscore}const{\isachardot}simps\ zero{\isacharunderscore}tag{\isacharunderscore}const{\isacharunderscore}def{\isacharparenright}\isanewline
\isacommand{qed}\isamarkupfalse%
%
\endisatagproof
{\isafoldproof}%
%
\isadelimproof
%
\endisadelimproof
\isanewline
\isanewline
\isacommand{end}\isamarkupfalse%
%
\begin{isamarkuptext}%
For comparing dates (which are represented by tags) on clocks, we need an order on tags.%
\end{isamarkuptext}\isamarkuptrue%
\isacommand{instantiation}\isamarkupfalse%
\ tag{\isacharunderscore}const\ {\isacharcolon}{\isacharcolon}\ {\isacharparenleft}order{\isacharparenright}order\isanewline
\isakeyword{begin}\isanewline
\ \ \isacommand{inductive}\isamarkupfalse%
\ less{\isacharunderscore}eq{\isacharunderscore}tag{\isacharunderscore}const\ {\isacharcolon}{\isacharcolon}\ {\isacartoucheopen}{\isacharprime}a\ tag{\isacharunderscore}const\ {\isasymRightarrow}\ {\isacharprime}a\ tag{\isacharunderscore}const\ {\isasymRightarrow}\ bool{\isacartoucheclose}\isanewline
\ \ \isakeyword{where}\isanewline
\ \ \ \ Int{\isacharunderscore}less{\isacharunderscore}eq{\isacharbrackleft}simp{\isacharbrackright}{\isacharcolon}\ \ \ \ \ \ {\isacartoucheopen}n\ {\isasymle}\ m\ {\isasymLongrightarrow}\ {\isacharparenleft}TConst\ n{\isacharparenright}\ {\isasymle}\ {\isacharparenleft}TConst\ m{\isacharparenright}{\isacartoucheclose}\isanewline
\isanewline
\ \ \isacommand{definition}\isamarkupfalse%
\ less{\isacharunderscore}tag{\isacharcolon}\ {\isacartoucheopen}{\isacharparenleft}x{\isacharcolon}{\isacharcolon}{\isacharprime}a\ tag{\isacharunderscore}const{\isacharparenright}\ {\isacharless}\ y\ {\isasymlongleftrightarrow}\ {\isacharparenleft}x\ {\isasymle}\ y{\isacharparenright}\ {\isasymand}\ {\isacharparenleft}x\ {\isasymnoteq}\ y{\isacharparenright}{\isacartoucheclose}\isanewline
\isanewline
\ \ \isacommand{instance}\isamarkupfalse%
%
\isadelimproof
\ %
\endisadelimproof
%
\isatagproof
\isacommand{proof}\isamarkupfalse%
\isanewline
\ \ \ \ \isacommand{show}\isamarkupfalse%
\ {\isacartoucheopen}{\isasymAnd}x\ y\ {\isacharcolon}{\isacharcolon}\ {\isacharprime}a\ tag{\isacharunderscore}const{\isachardot}\ {\isacharparenleft}x\ {\isacharless}\ y{\isacharparenright}\ {\isacharequal}\ {\isacharparenleft}x\ {\isasymle}\ y\ {\isasymand}\ {\isasymnot}\ y\ {\isasymle}\ x{\isacharparenright}{\isacartoucheclose}\isanewline
\ \ \ \ \ \ \isacommand{using}\isamarkupfalse%
\ less{\isacharunderscore}eq{\isacharunderscore}tag{\isacharunderscore}const{\isachardot}simps\ less{\isacharunderscore}tag\ \isacommand{by}\isamarkupfalse%
\ auto\isanewline
\ \ \isacommand{next}\isamarkupfalse%
\isanewline
\ \ \ \ \isacommand{fix}\isamarkupfalse%
\ x{\isacharcolon}{\isacharcolon}{\isacartoucheopen}{\isacharprime}a\ tag{\isacharunderscore}const{\isacartoucheclose}\isanewline
\ \ \ \ \isacommand{from}\isamarkupfalse%
\ tag{\isacharunderscore}const{\isachardot}exhaust\ \isacommand{obtain}\isamarkupfalse%
\ x\isactrlsub {\isadigit{0}}{\isacharcolon}{\isacharcolon}{\isacharprime}a\ \isakeyword{where}\ {\isacartoucheopen}x\ {\isacharequal}\ TConst\ x\isactrlsub {\isadigit{0}}{\isacartoucheclose}\ \isacommand{by}\isamarkupfalse%
\ blast\isanewline
\ \ \ \ \isacommand{with}\isamarkupfalse%
\ Int{\isacharunderscore}less{\isacharunderscore}eq\ \isacommand{show}\isamarkupfalse%
\ {\isacartoucheopen}x\ {\isasymle}\ x{\isacartoucheclose}\ \isacommand{by}\isamarkupfalse%
\ simp\isanewline
\ \ \isacommand{next}\isamarkupfalse%
\isanewline
\ \ \ \ \isacommand{show}\isamarkupfalse%
\ {\isacartoucheopen}{\isasymAnd}x\ y\ z\ \ {\isacharcolon}{\isacharcolon}\ {\isacharprime}a\ tag{\isacharunderscore}const{\isachardot}\ x\ {\isasymle}\ y\ {\isasymLongrightarrow}\ y\ {\isasymle}\ z\ {\isasymLongrightarrow}\ x\ {\isasymle}\ z{\isacartoucheclose}\isanewline
\ \ \ \ \ \ \isacommand{using}\isamarkupfalse%
\ less{\isacharunderscore}eq{\isacharunderscore}tag{\isacharunderscore}const{\isachardot}simps\ \isacommand{by}\isamarkupfalse%
\ auto\isanewline
\ \ \isacommand{next}\isamarkupfalse%
\isanewline
\ \ \ \ \isacommand{show}\isamarkupfalse%
\ {\isacartoucheopen}{\isasymAnd}x\ y\ \ {\isacharcolon}{\isacharcolon}\ {\isacharprime}a\ tag{\isacharunderscore}const{\isachardot}\ x\ {\isasymle}\ y\ {\isasymLongrightarrow}\ y\ {\isasymle}\ x\ {\isasymLongrightarrow}\ x\ {\isacharequal}\ y{\isacartoucheclose}\isanewline
\ \ \ \ \ \ \isacommand{using}\isamarkupfalse%
\ less{\isacharunderscore}eq{\isacharunderscore}tag{\isacharunderscore}const{\isachardot}simps\ \isacommand{by}\isamarkupfalse%
\ auto\isanewline
\ \ \isacommand{qed}\isamarkupfalse%
%
\endisatagproof
{\isafoldproof}%
%
\isadelimproof
%
\endisadelimproof
\isanewline
\isanewline
\isacommand{end}\isamarkupfalse%
%
\begin{isamarkuptext}%
For ensuring that time does never flow backwards, we need a total order on tags.%
\end{isamarkuptext}\isamarkuptrue%
\isacommand{instantiation}\isamarkupfalse%
\ tag{\isacharunderscore}const\ {\isacharcolon}{\isacharcolon}\ {\isacharparenleft}linorder{\isacharparenright}linorder\isanewline
\isakeyword{begin}\isanewline
\ \ \isacommand{instance}\isamarkupfalse%
%
\isadelimproof
\ %
\endisadelimproof
%
\isatagproof
\isacommand{proof}\isamarkupfalse%
\isanewline
\ \ \ \ \isacommand{fix}\isamarkupfalse%
\ x{\isacharcolon}{\isacharcolon}{\isacartoucheopen}{\isacharprime}a\ tag{\isacharunderscore}const{\isacartoucheclose}\ \isakeyword{and}\ y{\isacharcolon}{\isacharcolon}{\isacartoucheopen}{\isacharprime}a\ tag{\isacharunderscore}const{\isacartoucheclose}\isanewline
\ \ \ \ \isacommand{from}\isamarkupfalse%
\ tag{\isacharunderscore}const{\isachardot}exhaust\ \isacommand{obtain}\isamarkupfalse%
\ x\isactrlsub {\isadigit{0}}{\isacharcolon}{\isacharcolon}{\isacharprime}a\ \isakeyword{where}\ {\isacartoucheopen}x\ {\isacharequal}\ TConst\ x\isactrlsub {\isadigit{0}}{\isacartoucheclose}\ \isacommand{by}\isamarkupfalse%
\ blast\isanewline
\ \ \ \ \isacommand{moreover}\isamarkupfalse%
\ \isacommand{from}\isamarkupfalse%
\ tag{\isacharunderscore}const{\isachardot}exhaust\ \isacommand{obtain}\isamarkupfalse%
\ y\isactrlsub {\isadigit{0}}{\isacharcolon}{\isacharcolon}{\isacharprime}a\ \isakeyword{where}\ {\isacartoucheopen}y\ {\isacharequal}\ TConst\ y\isactrlsub {\isadigit{0}}{\isacartoucheclose}\ \isacommand{by}\isamarkupfalse%
\ blast\isanewline
\ \ \ \ \isacommand{ultimately}\isamarkupfalse%
\ \isacommand{show}\isamarkupfalse%
\ {\isacartoucheopen}x\ {\isasymle}\ y\ {\isasymor}\ y\ {\isasymle}\ x{\isacartoucheclose}\ \isacommand{using}\isamarkupfalse%
\ less{\isacharunderscore}eq{\isacharunderscore}tag{\isacharunderscore}const{\isachardot}simps\ \isacommand{by}\isamarkupfalse%
\ fastforce\isanewline
\ \ \isacommand{qed}\isamarkupfalse%
%
\endisatagproof
{\isafoldproof}%
%
\isadelimproof
%
\endisadelimproof
\isanewline
\isanewline
\isacommand{end}\isamarkupfalse%
\isanewline
%
\isadelimtheory
\isanewline
%
\endisadelimtheory
%
\isatagtheory
\isacommand{end}\isamarkupfalse%
%
\endisatagtheory
{\isafoldtheory}%
%
\isadelimtheory
%
\endisadelimtheory
%
\end{isabellebody}%
\endinput
%:%file=~/Documents/Recherche/Thesards/2014_Hai_NGUYEN_VAN/Heron_git/reiher2/reiher/src/TESL.thy%:%
%:%6=2%:%
%:%14=4%:%
%:%15=4%:%
%:%16=5%:%
%:%17=6%:%
%:%18=7%:%
%:%32=9%:%
%:%44=11%:%
%:%53=14%:%
%:%65=16%:%
%:%69=17%:%
%:%71=18%:%
%:%74=21%:%
%:%75=21%:%
%:%76=22%:%
%:%77=22%:%
%:%78=23%:%
%:%79=24%:%
%:%80=24%:%
%:%87=27%:%
%:%99=29%:%
%:%101=31%:%
%:%102=31%:%
%:%103=32%:%
%:%104=33%:%
%:%105=34%:%
%:%106=35%:%
%:%107=36%:%
%:%108=37%:%
%:%109=38%:%
%:%110=39%:%
%:%111=40%:%
%:%112=41%:%
%:%114=44%:%
%:%115=45%:%
%:%117=47%:%
%:%118=47%:%
%:%120=50%:%
%:%121=51%:%
%:%123=53%:%
%:%124=53%:%
%:%125=54%:%
%:%126=55%:%
%:%128=58%:%
%:%130=60%:%
%:%131=60%:%
%:%132=61%:%
%:%133=62%:%
%:%142=66%:%
%:%154=68%:%
%:%155=69%:%
%:%156=70%:%
%:%158=72%:%
%:%159=72%:%
%:%160=73%:%
%:%161=74%:%
%:%162=74%:%
%:%163=75%:%
%:%164=76%:%
%:%165=77%:%
%:%166=77%:%
%:%167=78%:%
%:%168=79%:%
%:%169=80%:%
%:%170=80%:%
%:%171=81%:%
%:%172=82%:%
%:%173=83%:%
%:%174=83%:%
%:%175=84%:%
%:%176=85%:%
%:%177=86%:%
%:%178=86%:%
%:%179=87%:%
%:%180=88%:%
%:%181=88%:%
%:%182=89%:%
%:%183=90%:%
%:%184=91%:%
%:%185=91%:%
%:%186=92%:%
%:%187=93%:%
%:%188=93%:%
%:%189=94%:%
%:%190=95%:%
%:%191=96%:%
%:%194=96%:%
%:%198=96%:%
%:%201=97%:%
%:%203=98%:%
%:%204=98%:%
%:%205=99%:%
%:%206=100%:%
%:%207=100%:%
%:%208=101%:%
%:%209=101%:%
%:%210=101%:%
%:%211=102%:%
%:%212=102%:%
%:%213=103%:%
%:%214=103%:%
%:%215=104%:%
%:%218=105%:%
%:%220=106%:%
%:%221=106%:%
%:%222=107%:%
%:%223=107%:%
%:%224=107%:%
%:%225=107%:%
%:%226=108%:%
%:%227=108%:%
%:%228=109%:%
%:%229=109%:%
%:%230=110%:%
%:%233=111%:%
%:%235=112%:%
%:%236=112%:%
%:%237=113%:%
%:%238=113%:%
%:%239=113%:%
%:%240=113%:%
%:%241=114%:%
%:%242=114%:%
%:%243=115%:%
%:%244=115%:%
%:%245=116%:%
%:%248=117%:%
%:%250=118%:%
%:%251=118%:%
%:%252=119%:%
%:%253=120%:%
%:%254=120%:%
%:%255=121%:%
%:%256=121%:%
%:%257=121%:%
%:%258=122%:%
%:%259=122%:%
%:%260=123%:%
%:%261=123%:%
%:%262=124%:%
%:%265=125%:%
%:%267=126%:%
%:%268=126%:%
%:%269=127%:%
%:%270=127%:%
%:%271=127%:%
%:%272=127%:%
%:%273=128%:%
%:%274=128%:%
%:%275=129%:%
%:%276=129%:%
%:%277=130%:%
%:%280=131%:%
%:%282=132%:%
%:%283=132%:%
%:%284=133%:%
%:%285=133%:%
%:%286=133%:%
%:%287=133%:%
%:%288=134%:%
%:%289=134%:%
%:%290=135%:%
%:%291=135%:%
%:%292=136%:%
%:%295=137%:%
%:%297=138%:%
%:%298=138%:%
%:%299=139%:%
%:%300=139%:%
%:%301=139%:%
%:%302=139%:%
%:%303=140%:%
%:%304=140%:%
%:%305=141%:%
%:%306=141%:%
%:%307=142%:%
%:%308=143%:%
%:%309=144%:%
%:%312=145%:%
%:%314=146%:%
%:%315=146%:%
%:%316=147%:%
%:%317=147%:%
%:%318=147%:%
%:%319=147%:%
%:%320=148%:%
%:%321=148%:%
%:%322=149%:%
%:%323=149%:%
%:%324=150%:%
%:%325=151%:%
%:%326=152%:%
%:%329=153%:%
%:%331=154%:%
%:%332=154%:%
%:%333=155%:%
%:%334=156%:%
%:%335=156%:%
%:%336=157%:%
%:%337=157%:%
%:%338=157%:%
%:%339=158%:%
%:%340=158%:%
%:%341=159%:%
%:%342=159%:%
%:%343=160%:%
%:%344=161%:%
%:%345=162%:%
%:%348=163%:%
%:%350=164%:%
%:%351=164%:%
%:%352=165%:%
%:%353=165%:%
%:%354=166%:%
%:%357=167%:%
%:%359=168%:%
%:%360=168%:%
%:%361=168%:%
%:%362=169%:%
%:%363=169%:%
%:%364=169%:%
%:%365=169%:%
%:%366=170%:%
%:%367=170%:%
%:%368=170%:%
%:%369=170%:%
%:%370=170%:%
%:%371=171%:%
%:%372=171%:%
%:%373=171%:%
%:%374=172%:%
%:%375=172%:%
%:%376=173%:%
%:%377=174%:%
%:%378=175%:%
%:%381=176%:%
%:%383=177%:%
%:%384=177%:%
%:%385=178%:%
%:%386=178%:%
%:%387=178%:%
%:%388=178%:%
%:%389=179%:%
%:%390=179%:%
%:%391=180%:%
%:%392=180%:%
%:%393=181%:%
%:%394=182%:%
%:%395=183%:%
%:%396=184%:%
%:%399=185%:%
%:%401=186%:%
%:%402=186%:%
%:%403=187%:%
%:%404=187%:%
%:%405=188%:%
%:%413=188%:%
%:%414=189%:%
%:%415=190%:%
%:%418=193%:%
%:%420=196%:%
%:%421=196%:%
%:%422=197%:%
%:%423=198%:%
%:%424=198%:%
%:%425=199%:%
%:%426=200%:%
%:%427=201%:%
%:%428=202%:%
%:%429=202%:%
%:%430=203%:%
%:%431=204%:%
%:%434=204%:%
%:%438=204%:%
%:%439=204%:%
%:%440=205%:%
%:%441=205%:%
%:%442=206%:%
%:%443=206%:%
%:%444=206%:%
%:%445=207%:%
%:%446=207%:%
%:%447=208%:%
%:%448=208%:%
%:%449=209%:%
%:%450=209%:%
%:%451=209%:%
%:%452=209%:%
%:%453=210%:%
%:%454=210%:%
%:%455=210%:%
%:%456=210%:%
%:%457=211%:%
%:%458=211%:%
%:%459=212%:%
%:%460=212%:%
%:%461=213%:%
%:%462=213%:%
%:%463=213%:%
%:%464=214%:%
%:%465=214%:%
%:%466=215%:%
%:%467=215%:%
%:%468=216%:%
%:%469=216%:%
%:%470=216%:%
%:%471=217%:%
%:%479=217%:%
%:%480=218%:%
%:%481=219%:%
%:%484=222%:%
%:%486=224%:%
%:%487=224%:%
%:%488=225%:%
%:%489=226%:%
%:%492=226%:%
%:%496=226%:%
%:%497=226%:%
%:%498=227%:%
%:%499=227%:%
%:%500=228%:%
%:%501=228%:%
%:%502=228%:%
%:%503=228%:%
%:%504=229%:%
%:%505=229%:%
%:%506=229%:%
%:%507=229%:%
%:%508=229%:%
%:%509=230%:%
%:%510=230%:%
%:%511=230%:%
%:%512=230%:%
%:%513=230%:%
%:%514=231%:%
%:%522=231%:%
%:%523=232%:%
%:%524=233%:%
%:%525=233%:%
%:%528=234%:%
%:%533=235%:%